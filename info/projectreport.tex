\documentclass[11pt]{article}
\usepackage{amsmath,amssymb,amsthm,bm,enumitem,hyperref,graphicx}
\usepackage[margin=1in]{geometry}
\title{Trajectory-Aware PAC-Bayes Bounds: A Gentle Introduction and Project Plan}
\author{Team Name}
\date{}

\newtheorem{theorem}{Theorem}[section]
\newtheorem{lemma}{Lemma}[section]
\newtheorem{definition}{Definition}[section]
\newtheorem{assumption}{Assumption}[section]

\begin{document}
\maketitle

\begin{abstract}
This project investigates trajectory-aware PAC-Bayes bounds for high-dimensional models. 
We start from binary classification and empirical risk minimization, motivate the need for generalization bounds, and gradually introduce PAC-Bayes theory in an accessible manner. 
The project focuses on a Gaussian posterior decomposition into top-subspace alignment, residual complexity, and trajectory curvature, together with stochastic control of trace and log-determinant approximations. 
We also provide an implementation plan and describe datasets for future experiments. 
Empirical results are left as placeholders.
\end{abstract}

\section{Introduction}
\subsection{What is Machine Learning?}
Machine learning aims to learn patterns from data to make predictions on unseen examples. 
A canonical example is binary classification: given data $\{(x_i,y_i)\}_{i=1}^n$ with $y_i \in \{0,1\}$, we want a function $f$ that predicts $y$ from $x$ as accurately as possible.

\subsection{Empirical Risk Minimization (ERM)}
Define a loss function $\ell(f(x),y)$ measuring prediction error. 
The \emph{empirical risk} is the average loss on the training set:
\[
R_S(f) = \frac{1}{n}\sum_{i=1}^n \ell(f(x_i),y_i).
\]
ERM chooses the function $f$ minimizing $R_S(f)$:
\[
\hat f = \arg\min_{f\in \mathcal{F}} R_S(f).
\]

\subsection{Generalization and the Need for Bounds}
Minimizing training error does not guarantee low error on new data:
\[
R(f) = \mathbb{E}_{(x,y)\sim \mathcal{D}}[\ell(f(x),y)].
\]
We need \emph{generalization bounds} that quantify how close $R_S(f)$ is to $R(f)$. 
Classical tools include VC dimension, Rademacher complexity, and concentration inequalities.

\section{PAC-Bayes Intuition}
\subsection{From Deterministic to Randomized Predictors}
Instead of choosing a single predictor $f$, PAC-Bayes considers a \emph{distribution over predictors}, called the \emph{posterior} $Q$. 
We measure the expected risk under $Q$:
\[
R(Q) = \mathbb{E}_{f\sim Q}[R(f)].
\]
PAC-Bayes bounds relate $R(Q)$ to the empirical risk $R_S(Q)$ plus a term depending on the divergence between $Q$ and a \emph{prior} $P$.

\subsection{The PAC-Bayes Bound (Simplified Form)}
For any prior $P$ and posterior $Q$, with probability $1-\delta$ over the sample:
\[
R(Q) \le R_S(Q) + \sqrt{\frac{\mathrm{KL}(Q\|P)+\log\frac{1}{\delta}}{2n}}.
\]
Intuition:
\begin{itemize}[noitemsep]
\item $R_S(Q)$ measures fit to training data.
\item $\mathrm{KL}(Q\|P)$ penalizes deviation from prior beliefs.
\item More data ($n$ larger) tightens the bound.
\end{itemize}

\section{Gaussian Posteriors and KL Decomposition}
\subsection{Gaussian Posterior}
Suppose $P = \mathcal{N}(\mu_P,\Sigma_P)$ and $Q = \mathcal{N}(\mu_Q,\Sigma_Q)$. 
The KL divergence can be written explicitly:
\[
\mathrm{KL}(Q\|P) = \frac{1}{2}\Big[ \mathrm{tr}(\Sigma_P^{-1}\Sigma_Q) + (\mu_P-\mu_Q)^\top \Sigma_P^{-1}(\mu_P-\mu_Q) - d + \log\frac{\det\Sigma_P}{\det\Sigma_Q}\Big].
\]

\subsection{Top-Subspace and Residual Decomposition}
Let $V_r$ be the top-$r$ eigenvectors of $\Sigma_P$, and $\Pi_r = V_r V_r^\top$. Then:
\begin{align*}
\text{Top-subspace alignment: } & A_r = \|\Pi_r (\mu_Q-\mu_P)\|_{\Sigma_P^{-1}}^2,\\
\text{Residual complexity: } & R_r = \mathrm{tr}(\Sigma_P^{-1}\Sigma_Q) - \mathrm{tr}(\Pi_r \Sigma_P^{-1}\Sigma_Q \Pi_r).
\end{align*}
This decomposition separates dominant directions from smaller variance directions, improving interpretability.

\section{Trajectory-Averaged Curvature}
\subsection{Motivation}
When learning with stochastic optimization (e.g., SGD), the model parameters evolve along a trajectory $\{\hat w_t\}_{t=1}^T$. 
Local curvature information from Hessians can refine the posterior:
\[
H_t = \nabla^2 R_S(\hat w_t).
\]

\subsection{Trajectory-Averaged Hessian}
Define the trajectory-averaged Hessian:
\[
\bar H_T = \frac{1}{T}\sum_{t=1}^T H_t.
\]
This improves stability and captures typical curvature along optimization.

\subsection{Posterior Covariance with Scaling}
We define
\[
\Sigma_Q = c_Q^2 (\bar H_T + \lambda I)^{-1},
\]
where $c_Q$ is a scaling constant, $\lambda>0$ ensures numerical stability.

\section{Stochastic Trace and Log-Determinant Estimation}
\subsection{Hutchinson Estimator}
For large matrices, direct computation of $\mathrm{tr}(H)$ or $\log\det(H)$ is costly. Hutchinson's method estimates traces with random vectors $z_i\sim \mathcal{N}(0,I)$:
\[
\mathrm{tr}(H) \approx \frac{1}{m}\sum_{i=1}^m z_i^\top H z_i.
\]

\subsection{Heuristic Log-Det Approximation}
Similarly,
\[
\mathrm{tr}\log(I+\lambda^{-1} H) \approx \frac{1}{m}\sum_{i=1}^m z_i^\top \log(I+\lambda^{-1} H) z_i.
\]
This provides a practical method to compute curvature contributions to the KL decomposition.

\section{Trajectory-Aware PAC-Bayes Bound (Summary)}
Combining the above:
\[
R(Q) \le R_S(Q) + \sqrt{\frac{\mathrm{KL}(Q\|P) + \epsilon + \log\frac{1}{\delta}}{2n}},
\]
where $\epsilon$ accounts for stochastic estimation errors in trace/log-det.

\section{Implementation Plan}
\begin{enumerate}[noitemsep]
\item Compute a parameter trajectory $\{\hat w_t\}$ using SGD.
\item Estimate Hessians $H_t$ along the trajectory.
\item Compute trajectory-averaged Hessian $\bar H_T$.
\item Define Gaussian posterior $Q$ with covariance $\Sigma_Q$.
\item Decompose KL into top-subspace, residual, and curvature terms.
\item Estimate stochastic terms with Hutchinson estimator.
\end{enumerate}

\section{Datasets (to be introduced)}
Planned datasets:
\begin{itemize}[noitemsep]
\item Synthetic binary classification dataset with known structure.
\item Real dataset: e.g., MNIST or UCI tabular dataset.
\end{itemize}
Empirical experiments, figures, and tables will be added after implementation.

\section{Discussion and Practical Guidance}
\begin{itemize}[noitemsep]
\item $c_Q$ balances KL tightness vs. stability.
\item Top-$r$ projection simplifies analysis in high dimensions.
\item Number of Hutchinson samples $m$ trades off computation and estimation error.
\item Trajectory length $T$ affects stability; independence assumption is heuristic.
\item The bound is valid for smooth losses; non-smooth losses require alternative curvature approximations.
\end{itemize}

\section{References}
\bibliographystyle{plain}
\bibliography{references}

\end{document}